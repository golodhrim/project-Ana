\startenvironment env_analysis

\enableregime[utf]
\setuplanguage[deo][leftquote=\uperleftsinglesixquote,rightquote=\uperrightsixquote,leftquotation=\leftguillemot,rightquotation=\rightguillemot]

\usemodule[greek]
\usemodule[units]
\usemodule[gnuplot]
\usemodule[newmat]
\usemodule[amsl]
\usesymbols[was]
\usemodule[vim]

\setupcolors[rgb=no,cmyk=yes,hex,spot=yes,state=start,overprint=yes]
\setupinteraction[state=start]

\usemodule[tikz]
\usemodule[pgfplots]
\usetikzlibrary[calc,through,backgrounds,matrix,arrows]

\usemodule[bib]
\usemodule[bibltx]
%\setuppublications[alternative=apa]
\setuppublications[state=start,refcommand=authoryears]
\setupbibtex[database=analysis]
\setupcite[authoryears][compress=no,left=,right=]
%\setupcite[authoryears][left=,right=]

\setuppapersize[A4][A4]

\setuppagenumbering[alternative=doublesided,	
			location={footer,marginedge},
			way=bytext, sectionnumber=yes, partnumber=yes]

\setuplayout[location=middle,
	topspace=3cm,
	header=14pt,
	headerdistance=10pt,
	width=middle,
	cutspace=4.5cm,
	rightmargindistance=0.4cm,
	leftmargindistance=0.2cm,
	backspace=1.2cm,
	height=fit,
	rightmargin=4cm,
	leftmargin=1cm,
	bottomspace=1cm,
	footer=0.8cm,
	setup=strict]
% all setups...

%\showgrid
%\showframe
%\showsetups
%\showlayout

\setupheadertexts[][chapter][part][]
\setupbackgrounds[header][text][frame=off,bottomframe=on]
\setuptolerance[{horizontal, verytolerant},{vertical, verystrict}]
\clubpenalty=8000		% no clubs / keine Schusterjungen
\widowpenalty=8000		% no widows / keine Hurenkinder
\tolerance=1750		% line breaking tolerance (ca. 500--3000)
\emergencystretch=10 pt	% additional stretching space per line
\setupwhitespace[medium]
\setupfootnotes[conversion=set 2, way=bypage]

\definepagebreak[chapter][yes,footer,right]

\def\MyChapterCommand#1#2%
	{\framed[frame=off,bottomframe=on,topframe=on]
		{\vbox{\headtext{chapter} #1\blank#2}}
	}

\setuphead[chapter][command=\MyChapterCommand,resetnumber=no]
\setupheadtext[chapter=Kapitel]

\setupsection[section][previousnumber=no,number=yes]

\def\MySectionCommand#1#2%
	{§ #1 - #2}
\setuphead[section][command=\MySectionCommand]

\def\MySubjectCommand#1#2%
	{#1#2}
\setuphead[subject][command=\MySubjectCommand]

\setuphead[subsection][number=yes,style=\ss]
\def\MySubSubSectionCommand#1#2%
	{#1 #2}
\setuphead[subsubsection][command=\MySubSubSectionCommand,style=slanted,number=yes,before={\blank[big]}, after=\nowhitespace]

\setupfloats[spaceafter=3*medium]

\setuptables[bodyfont=small]
\setupcaption[table][style={\ssx\setupinterlinespace[line=2.5ex]},align=left]

\setupfloat[table][criterium=0.5\textwidth,default=inner]
\setupcaption[figure][style={\ssx\setupinterlinespace[line=2.5ex]},align=left]

\definefloat[Bigfigure][Bigfigures][figure]
\setupfloat[Bigfigure][location=inner]

\definefloat[Bigtable][Bigtables][table]
\setupfloat[Bigtable][location=inner]

\definefloat[vignette][figure]
\setupfloat[vignette][leftmargindistance=-\outermargintotal,rightmargindistance=-outermargintotal,default={outer,none,low}]

\setupinmargin[align=right,style=\ss\tfx\setupinterlinespace]

\setupitemize[align=right]

\startsectionblockenvironment[appendix]

\setuptabulate[align=right,split=yes]

\definepagebreak[chapter][yes,footer,right]
\setuphead[chapter][bodypartconversion=Character,page=chapter]
\setuphead[section][number=yes]
\setuphead[subsection][number=yes]
\setuphead[subsubsection][number=yes]
\definefloat[BigFramedFigure][BigFramedFigures][figure]
\setupfloat[BigFramedFigure][location=inner,frame=on,offset=2pt]
\definefloat[FramedFigure][FramedFigures][figure]
\setupfloat[FramedFigure][frame=on,offset=2pt]
\stopsectionblockenvironment

\setupregister[index][imp][pagestyle=bold]
\setupcombinedlist[content][level=3,alternative=c]
\setuplist[chapter][style=bold]

\definestructureconversionset[frontpart:pagenumber][][romannumerals]
\definestructureconversionset[bodypart:pagenumber] [][numbers]

\setuplist[chapter][pageconversionset=pagenumber]

\startstructureblockenvironment[frontpart]
\setupuserpagenumber[numberconversion=romannumerals]
\setuppagenumber[number=1]
\stopstructureblockenvironment

\startstructureblockenvironment[bodypart]
\setuppagenumber[number=1]
\stopstructureblockenvironment

%% Descriptionvariablen
%
\definedescription[descr][headstyle=bold, style=normal, location=hanging, width=fit, margin=1cm]


%% Umgebungsvariablen
%
\defineenumeration[bem][text={Bemerkung }, location=serried, width=fit, identing=first, distance=1em]
\defineenumeration[frage][text={\underbar{Frage:} }, location=serried, width=fit, identing=first, distance=0.5em]
\setupnumber[frage][number=no]
\defineenumeration[proof][text={\bs Beweis }, location=serried, width=fit, ident=yes, distance=1em, way=bytext, closesymbol=\QED]
\setupnumber[proof][number=no]
\defineenumeration[satz][text={\bf Satz }, location=serried, width=fit, identing=first, distance=1em]
\defineenumeration[bsp][text={Beispiel }, location=serried, width=fit, identing=first, distance=0.5em]
\setupnumber[bsp][number=no]
\defineenumeration[defn][text={Definition }, location=serried, width=fit, identing=first, distance=1em]
\defineenumeration[lemma][text={Lemma }, location=serried, width=fit, identing=first, distance=1em]
\defineenumeration[corollary][text={Corollary }, location=serried, width=fit, identing=first, distance=1em]
\defineenumeration[nota][text={Notation }, location=serried, width=fit, identing=first, distance=1em]
\setupnumber[nota][number=no]
\defineenumeration[usage][text={Anwendung }, location=serried, width=fit, identing=first, distance=1em]
\setupnumber[usage][number=no]
\defineenumeration[prinzip][text={Prinzip }, location=serried, width=fit, identing=first, distance=1em]
\setupnumber[prinzip][number=no]
\defineenumeration[eig][text={Eigenschaften }, location=serried, width=fit, identing=first, distance=1em]
\setupnumber[eig][number=no]
\defineenumeration[folg][text={Folgerung }, location=serried, width=fit, identing=first, distance=1em]
\setupnumber[folg][number=no]

%% Code Highligthing with vim
%
\definevimtyping[SAGE][syntax=python,alternative=pscolor]

%% Mathematische Mengen
%
\def\C{\complexes}
\def\R{\reals}
\def\Q{\rationals}
\def\Z{\integers}
\def\N{\naturalnumbers}
\def\P{\primes}
% Weitere Doppelstroke Zeichen mit \def\<TYPE>{\blackboard{<TYPE>}} definieren
\def\F{\blackboard{F}}
\def\Fam{{\cal F}}
\def\O{{\cal O}}

%% Hilfreiche Operatoren
%
\def\stackrelo#1#2{\mathrel{\mathop{#1}\limits^{\small #2}}}
\def\stackrelu#1#2{\mathrel{\mathop{#1}\limits^{\small #2}}}

%% sonstige mathem. Operatoren
%
\def\abs#1{\left| #1 \right|}
\def\norm#1{\left|\left| #1 \right|\right|}
\def\ord{{\rm ord}}
\def\Log{{\rm Log}}
\def\ZZ{$Z\mkern-16mu_{\textstyle Z}\mkern5mu$}
\def\Res{{\rm Res}}
\def\div{{\rm div}}
\def\dim{{\rm dim}}
\def\core{{\rm kern}}
\def\bild{{\rm bild}}
\def\grad{{\rm grad}}

%% mathem. Symbole
%
\def\lightning{\symbol[wasy general][lightning]}
\def\iddots{\mirror{\ddots}}
%% URLS including
%
\input urls.tex

%% Warnungssymbol
%
\def\warning{\inmargin{\reuseMPgraphic{achtung}}\underbar{Vorsicht:}}
\startreusableMPgraphic{achtung}
a = sqrt(400-25)+5;
path d, s, l, p;
d := ( (0,5)--(20,5)--(10,a)--cycle )  ;
s := ( (10,5)--(10,0) ) ;
l := ( (10,18)--(10,11) ) ;
p := fullcircle scaled (0.5) shifted (10,9) ; 
draw p;
draw l;
draw s;
draw d withcolor .625 red;
\stopreusableMPgraphic

\def\notice{\inmargin{\reuseMPgraphic{hinweis}}\underbar{Hinweis: }}
\startreusableMPgraphic{hinweis}
path l, p, b;
b:= fullcircle scaled (0) shifted (0,0);
l:= ( (20,10)--(20,3) );
p:= fullcircle scaled (0.5) shifted (20.25,1) ;
draw b withcolor white;
draw p withcolor .625 red;
draw l withcolor .625 red;
\stopreusableMPgraphic

%% Coverpage Setup
%
\startuseMPgraphic{cover}
	StartPage ;
		color bordeauxred ; bordeauxred := cmyk (0.34,0.92,0.64,0.34) ;
		color darkblue; darkblue := cmyk(1,0.99,0,0.32) ;
		fill Page withcolor darkblue ;
		z1 = (0cm, 7.0112cm) ;
		z2 = (21cm, 7.0112cm) ;
		z3 = (21cm,11.3444cm) ;
		z4 = (0cm,11.3444cm) ;
		path p ; p := z1--z2--z3--z4--cycle ;
		draw p withcolor bordeauxred ;
		fill p withcolor bordeauxred ;
	StopPage;
\stopuseMPgraphic

\defineoverlay[cover][\useMPgraphic{cover}]

%% ColorDefinition
%
\definecolor[bordeauxred][c=.34, m=.92, y=.64, k=.34]
\definecolor[darkblue][c=1,m=.99,y=0,k=.32]

\stopenvironment
